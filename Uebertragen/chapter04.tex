\chapter{Vorgehen}%
\label{chap:GrundlagenundMethoden}%

\tikzstyle{diamond} = [draw, shape aspect=2.5, shape=diamond, minimum width=3cm, minimum height=0.8cm, text centered, align=center]
\tikzstyle{rectangle} = [draw, shape=rectangle, minimum width=3cm, minimum height=0.8cm, text centered, align=center]
\tikzstyle{arrow} = [thick,->,>=Stealth, arrowlabel/.style={right}]

Dieses Kapitel beschreibt die beiden Messsysteme und wie die gewonnenen Daten genutzt werden, um zum einen die
Impedanz des Zyklisierer aus der

Zunächst wird das System zur Untersuchung der Einflüsse von Fomrierungseffekten auf die Zellimpedanz.
Als Erstes wird auf die verwendete Hardware und Software eingegangen.
Im Anschluss werden die eingesetzen Ladeprotokolle und der Prüfplan erklärt.

Danach wird das Setup von BaSyTec beschrieben.
Hier wird ebenfalls zunächst auf Hardware und Software eingegangen, bevor der Prüfplan erläutert wird.

\section{Untersuchung der Formierung}
\label{sec:form}
Die Zyklisieren und Prüfen der Zellen erfolgt mit einer Biologic-Einheit, die über einen Laborcomputer mit der Testsoftware verbunden ist.
Auf diesem Rechner werden alle Messdaten geplant, gestartet und gespeichert.
Nach Abschluss der Prüfung werden die Ergebnisse auf das NAS-Laufwerk des Lehrstuhls übertragen.
Von dort lassen sich die Dateien über eine Streamlit-WebApp verarbeiten, darstellen und in eine postgresql-Datenbank
speichern.

\subsection{Hardware}
\label{subsec:hardware-bio}
Im folgenden Abschnitt wird die verwendete Hardware des Biologic-Prüfaufbaus beschrieben.
\subsubsection{Zelle}
Zur Untersuchung der Formierung werden neue Sony US18650VTC5A verwendet.
Die Eigenschaften dieser Zellen sind in \autoref{tab: sony-zelle} aufgeführt.
\begin{table}[h]
    \centering
    \begin{tabular}{lc}
        \toprule
        Parameter & Wert \\
        \midrule
        Kapazität & \SI{2500}{\milli\ampere\hour} \\
        Ladespannung & \SI{4.2}{\volt} \\
        untere Abschaltspannung & \SI{2.5}{\volt} \\
        Max. Ladestrom & \SI{6}{\ampere} \\
        Max. Entladestrom & \SI{35}{\ampere} \\
        Temperaturbereich Laden & \SIrange{0}{45}{\celsius} \\
        Temperaturbereich Entladen & \SIrange{-20}{60}{\celsius} \\
        \bottomrule
    \end{tabular}
    \caption{Eigenschaften der Zellen vom Typ Sony US18650VTC5A}
    \label{tab: sony-zelle}
\end{table}

\begin{figure} [H]
    \centering
    \includegraphics[width=0.4\linewidth]{figures/hardware/Sony-VTC5A-2}
    \caption{Sony US18650VTC5A~\cite{sony_zelle}}
    \label{fig:sony-pic}
\end{figure}

\subsubsection{Klimakammer}
Aufgrund der temperaturabhängigen Eigenschaften von Batteriezellen ist es wichtig die Zellen während der Versuche
in einer Klimakammer zu betrieben.
Dies gewährleistet eine konstante Umgebungstemperatur während der gesamten Prüfungsdauer.
Die kontrollierten Bedingungen reduzieren externe Einflüsse auf die Messergebnisse und ermöglichen eine
reproduzierbare Durchführung der Prüfungen.
Dadurch lassen sich Veränderungen der Zellparameter klar der getesteten Belastung zuordnen.
Als Klimakammer wird eine Binder KB-400 verwendet.

\begin{figure} [H]
    \centering
    \includegraphics[width=0.4\linewidth]{figures/hardware/binder}
    \caption{Binder KB-400 Klimakammer~\cite{binder_kammer}}
    \label{fig:binder}
\end{figure}

\subsection{Software}
\label{subsec:software-bio}
Als Testsoftware wird die

\subsection{Prüfplan}
\label{subsec:pruefplan-bio}
Der Prüfplan orientiert sich an einer bestehenden Messreihe und wurde für diese Untersuchung übernommen, wobei nur
die Ladeparameter gezielt abgeändert werden, um deren Einfluss auf die Impedanz zu analysieren.
Die Versuche werden bei einer Temperatur von \SI{25}{\celsius} durchgeführt.
Es werden drei Ladebereiche des \gls{soc} betrachtet: ein enger Bereich von \SIrange{20}{80}{\percent}, ein
erweiterter Bereich von \SIrange{10}{90}{\percent} sowie eine vollständige Ausnutzung des Ladebereichs von
\SIrange{0}{100}{\percent}.
Für jeden Ladebereich werden drei verschiedene C-Raten angewendet: \SI{0,5}{C}, \SI{1}{C} und \SI{2}{C}.
Zur Untersuchen des Einflusses der Temperatur wird zusätzlich ein weiterer Versuch bei einer erhöhten Temperatur von
\SI{50}{\celsius} durchgeführt.
Dabei kommt eine C-Rate von \SI{1}{C} und ein \gls{soc}-Bereich von \SIrange{10}{90}{\percent} zum Einsatz.

\begin{table}[h]
    \centering
    \begin{tabular}{lccc}
        \toprule
        Zelle     & Strom (C-Rate) &  Ladebereich (\% SoC)       & Temperatur (\si{\degreeCelsius}) \\
        \midrule
        JT\_001   & \SI{1.0}{C}    & 10–90                & 50 \\
        JT\_002   & \SI{0.5}{C}    & 0–100                & 25 \\
        JT\_003   & \SI{2.0}{C}    & 20–80                & 25 \\
        JT\_004   & \SI{2.0}{C}    & 10–90                & 25 \\
        JT\_005   & \SI{2.0}{C}    & 0–100                & 25 \\
        JT\_006   & \SI{1.0}{C}    & 20–80                & 25 \\
        JT\_007   & \SI{1.0}{C}    & 10–90                & 25 \\
        JT\_008   & \SI{1.0}{C}    & 0–100                & 25 \\
        JT\_009   & \SI{0.5}{C}    & 10–90                & 25 \\
        JT\_010   & \SI{0.5}{C}    & 20–80                & 25 \\
        \bottomrule
    \end{tabular}
    \caption{Übersicht der verwendeten Zellen und Ladeparameter}
    \label{tab:ladeparameter_gesamt}
\end{table}

Nach einer Initialen Kapazitätsmessung wird die Impedanz der Zelle auf 15 verschiedenen \glspl{soc} zwischen
\SI{250}{\milli\ampere\hour} und \SI{2000}{\milli\ampere\hour} in einem Frequenzbereich von \SIrange{0,2}{5000}{\hertz} bestimmt.
Außerdem werden auf jedem \gls{soc}-Niveau \gls{deis}-Messungen bei Ladeströmen zwischen \SI{0,25}{C} und \SI{4}{C}
und einer Frequenz von \SI{200}{\hertz} durchgeführt.
Vor jeder \gls{eis}-Messung werden die Zellen zunächst \SI{4}{\hour} ruhen gelassen, um ein thermisches und
elektrochemisches Gleichgewicht herzustellen.
Nach der \gls{eis}-Messung ruhen die Zellen ebenfalls für \SI{30}{\minute}.
Im Anschluss werden die Zellen mit unterschiedlichen Parametern fünfmal zyklisch mit \gls{cccv} ge- und entladen, bevor
wieder eine
Impedanzprüfung durchgeführt wird.
Nach 25 Zyklen wird noch einmal eine Kapazitätsmessung und eine \gls{ocv}-Messung bei \SI{0,05}{C}
mit \gls{cc} durchgeführt.
Die \gls{ocv} wird verwendet um Eigenschaften der Elektroden im Verlauf der Zyklisierung zu untersuchen.
Danach wird der Prüfplan erneut gestartet.

\begin{figure} [h]
    \hspace{3cm}% Abstand nach rechts verschieben
    \centering
    \begin{tikzpicture}[node distance=0.75cm and 5]

    % Nodes
    \node (start) [rectangle] {Kapazitätsmessung};
    \node (eis) [rectangle,below=of start] {EIS und DEIS Prüfung};
    \node (ageing1) [diamond, below=of eis] {5 Zyklen};
    \node (kapa) [rectangle, below=of ageing1] {Kapazitätsmessung};
    \node (ocv) [rectangle, below=of kapa] {OCV-Messung};

    % Arrows
    \draw [arrow] (start) -- (eis);
    \draw [arrow] (eis) -- (ageing1);
    \draw [arrow] (ageing1) to[out=0, in=0, looseness=1.5] node[arrowlabel] {4-Mal wiederholen} (eis);
    \draw [arrow] (ageing1) --node[arrowlabel] {nach 25 Zyklen} (kapa);
    \draw [arrow] (kapa) -- (ocv);

    \end{tikzpicture}
    \caption{Prüfplan für die Untersuchung von Formierungseffekten}\label{fig:pruefplan-bio}
\end{figure}

\section{Untersuchung des Impedanz von Basytec}
\label{sec:basytec}
Mithilfe von Multi-Sine-Anregung kann die dauer einer \gls{eis}-Messung stark reduziert werden.
Für stationäre Messungen ist die Dauer der Messung nicht besonders relevant, da sich die Eigenschaften der Zelle während
dieser Messungen nur gering verändern.
Im dynamischen Fall ist es jedoch wichtig die Dauer der Messung möglichst gering zu halten, da sich die Zellen unter
Belastung durchaus stark erwärmen können.
Mit dem Zyklisierer von Biologic sind leider keine Multi-Sine-Anregungen möglich.
Daher wird für dynamische Messungen ein Setup von BaSyTec verwendet.
Auf diesem Setup werden dann durch ein Inspectrum.C-Messgerät der Firma Safion Impedanzmessungen durchgeführt.
Für die Untersuchung des Impedanzeinflusses des Prüfaufbaus werden mehrere \gls{eis}- und \gls{deis}-Messungen mit dem
Zyklisierer der Firmer Biologic durchgeführt.
Anschließend wird dieselbe Zelle in den Batteriezyklisierer der Firma BaSyTec überführt, wo die identischen
Messreihen mithilfe des Inspectrum.C-Messgerät durchgeführt werden.
Es ist wichtig, dass die Testbedingungen auf beiden Geräten identisch sind, um vergleichbare Messungen zu
gewährleisten und ausschließlich den Einfluss der Zyklisierer zu untersuchen.
Daher dürfen sich weder der \gls{soc} noch die Temperatur, die Frequenzen oder die Ströme, bei denen die Messungen
durchgeführt werden, ändern.
Die Untersuchung wird auf \SI{50}{\%} des \gls{soc} und bei \SI{25}{\degreeCelsius} durchgeführt.
Als Zelle wird eine Panasonic NCR18650PF verwendet, welche schon einige Zyklen durchlaufen ist, um Einflüsse der
Formierung zu vermeiden.

\subsection{Hardware}
\label{subsec:hardware-basy}
Im folgenden Abschnitt wird die für den Basytec-Prüfaufbau eingesetzte Hardware erläutert.

\subsubsection{Zelle}
Um Einflüsse der \gls{sei}-Bildung zu eliminieren wird eine bereits zyklisiserte Zelle verwendet.
Diese Zelle vom Typ Panasonic NCR18650PF\@.
Die Eigenschaften dieser Zellen sind in \autoref{tab: panasonic-zelle} aufgeführt.
\begin{table}[H]
    \centering
    \begin{tabular}{lc}
        \toprule
        Parameter & Wert \\
        \midrule
        Kapazität & \SI{2900}{\milli\ampere\hour} \\
        Ladespannung & \SI{4.2}{\volt} \\
        untere Abschaltspannung & \SI{2.5}{\volt} \\
        Max. Ladestrom & \SI{2}{\ampere} \\
        Max. Entladestrom & \SI{10}{\ampere} \\
        Temperaturbereich Laden & \SIrange{0}{60}{\celsius} \\
        Temperaturbereich Entladen & \SIrange{-20}{60}{\celsius} \\
        \bottomrule
    \end{tabular}
    \caption{Eigenschaften der Zellen vom Typ Panasonic NCR18650PF}
    \label{tab: panasonic-zelle}
\end{table}

\begin{figure} [H]
    \centering
    \includegraphics[width=0.4\linewidth]{figures/hardware/panasonic}
    \caption{Panasonic NCR18650PF~\cite{panasonic}}
    \label{fig:panasonic-pic}
\end{figure}



\subsubsection{Klimakammer}
Um die Impedanz des Zyklisierers zu untersuchen ist es noch wichtiger die Temperatur der Zelle konstant zu halten.
Daher befinden sich die Zellen im BaSyTec-Setup ebenfalls in einer Klimmakammer.
Hier kommt eine Klimakammer der Firma Mermet zum Einsatz.

\begin{figure} [H]
    \centering
    \includegraphics[width=0.4\linewidth]{figures/hardware/mermet}
    \caption{Mermet IP110eco Klimakammer~\cite{mermet_chamber}}
    \label{fig:mermet}
\end{figure}

\subsubsection{Zyklisierer}
Das Laden und Entladen auf diesem Setup erfolgt über ein BaSyTec-XCTS\@.
Die BaSyTec-Batterietestsoftware ermöglicht die Erstellung und Bearbeitung von Testplänen für den Zyklisierer.
In den Plänen können zusätzlich CAN-Parameter integriert werden, die zum Starten von Impedanzmessungen genutzt werden.
\begin{figure} [H]
    \centering
    \includegraphics[width=0.4\linewidth]{figures/hardware/BaSyTec_XCTS}
    \caption{BaSyTec XCTS~\cite{basytec}}
    \label{fig:xcts}
\end{figure}

\subsubsection{Impedanz-Messsystem}
Da der XCTS keine eigenständigen \gls{eis}-Messungen durchführen kann, wird das System Inspectrum.10-5 ES der
Firma Safion GmbH eingesetzt.
Es besteht aus einem Messgerät und einem Multiplexer.
Die Spannungsantwort wird über seperate Sense-Leitungen erfasst, wobei eine Vierleiter-Messung durchgeführt wird.
Das Messgerät erfasst die Impedanz mit einer Multisine-Anregung präzise an bis zu 32 Frequenzpunkten in einer Sekunde~\cite{inspectrum}.
Die Frequenzpunkte können manuell anhand einer linearen oder logarithmischen Verteilung des Spektrums ausgewählt werden.
Dadurch lässt sich die Messauflösung an die experimentellen Anforderungen anpassen.
Um die Frequenzpunkte festzulegen, muss ein Computer mit dem Messgerät verbunden werden.
Über den Computer können mit der Inspectrum Suite Software die Frequenzpunkte verändert werden.
Andere Parameter des Anregungssignals, wie die Amplitude ode Dauer der Anregung, könne ebenfalls angepasst werden.

Das Gerät kann über eine CAN-Schnittstelle konfiguriert, angesteuert und ausgelesen werden, wodurch eine einfache
integriert in übergeordnete Testinfrastrukturen möglich ist.
\begin{figure} [H]
    \centering
    \includegraphics[width=0.4\linewidth]{figures/hardware/inspectrum_meas}
    \caption{Inspectrum.10-5 ES 105~\cite{inspectrum}}
    \label{fig:inspectrum_meas}
\end{figure}
Mit dem Multiplexer lässt sich das Messsystem auf bis zu 20 Kanäle erweitern, was die Untersuchung mehrere Zellen mit
einem einzelnen Messgerät ermöglicht.
In dieser Arbeit wird die Kombination dieser beiden Geräte als Safion Einheit bezeichnet.
\begin{figure} [H]
    \centering
    \includegraphics[width=0.4\linewidth]{figures/hardware/inspectrum_mux}
    \caption{Inspectrum.10-5 MUX~\cite{inspectrum}}
    \label{fig:inspectrum_mux}
\end{figure}

\subsubsection{Raspberry Pi}
Ein Raspberry Pi ist ein kompakter aber leistungsfähiger Einplatinencomputer.
Er ist aufgrund seiner geringen Kosten und einfachen Handhabung eine hervorragende Plattform für eine
große Anzahl an Anwendungen~\cite{raspberry_rs485}.
Die kombination mit einem CAN-HAT ermöglicht es den Pis untereinander und dem Inspectrum zu kommunizieren.
Das System ist in einer Master-Slave-Architektur aufgebaut.
Ein Master-Raspberry-Pi koordiniert Anfragen der Slave-Raspberry-Pis und verhindert Konflikte zwischen Messungen.
In dem Setup werden mehrere Pis vom Typ Raspberry Pi3b+ verwendet, die in einem 3d-gedruckten Gehäuse angeordnet sind.
Jeder Pi ist mit einem adaptiven Lüfter verbunden, der das Überhitzen der Geräte verhindert.
\begin{figure} [H]
    \centering
    \includegraphics[width=0.4\linewidth]{figures/hardware/raspberry}
    \caption{Raspberry Pi mit CAN-Hat~\cite{raspberry_rs485}}
    \label{fig:raspberry}
\end{figure}

\subsection{Software}
\label{subsec:software-basy}
Zur Gewährleistung eines sicheren Kanalwechsels wurde ein Algorithmus entwickelt, der dem Safion fehlerfreie Messungen
auf verschiedenen Kanälen ermöglicht.
Ohne diesem Verfahren würde bei gleichzeitiger Messung ein Kanal den andere überschreiben.
Zur vermeidung solcher Konflikte übernimmt ein Master-Raspberry-Pi die Steuerung und Koordination der Anfragen mehrere
Slave-Raspberry-Pis.
Dabei blockiert der Master eine neue Messung sollte der Safion noch mit einer anderen beschäftigt sein.

Die Testsoftware von Basytec ermöglicht es im Prüfprogramm Signale über CAN-Bus zu senden.
Neben periodisch gesendeten Nachrichten die Daten wie Zeit, aktuelle Zellspannung, angelegten Strom, Temperatur
und geladene Kapazität enthalten wird zu Beginn des Prüfplans die Nennkapazität der Zelle übermittelt.
Vor jeder Messung wird die aktuelle C-Rate übertragen.
Zum Starten und Beenden von \gls{eis}-Messungen wird eine spezifische Nachricht gesendet.
Nachdem die Slave-Pis diese Nachricht empfangen haben, fragen sie beim Master-Pi an ob der Safion gerade mit einer
anderen Messung beschäftigt ist.
In diesem Fall erstellt der Master-Pi eine Liste der Slave-Pis, die eine Messung durchführen wollen, und
bearbeitet diese der Reihe nach ab.
Erhält ein Slave-Pi die Freigabe des Masters schickt er eine CAN-Nachrricht an den Safion, der daraufhin eine
\gls{eis}-Messung durchführt und die erhaltenen Daten über CAN-Bus an den Pi zurück übermittelt.
Der Pi verwendet die übermittelte Nennkapazität und aktuelle C-Rate, um die Zeit zu berechnen, die benötigt wird,
bis die Zelle \SI{0,5}{\%} geladen ist.
In dieses Zeitraums führt er fortlaufend Messungen, bis der Messvorgang beendet wird.
Nachdem er diese beendet hat, gibt er den Safion wieder frei und übermittelt dies dem Master-Pi.
Dieser übergibt die Freugabe zur Messung dann an den Nächsten Slave-Pi.

\subsection{Prüfplan}
\label{subsec:pruefplan-basy}
Um auf einen \gls{soc} von \SI{50}{\%} zu kommen, wird die Zelle erst vollkommen entladen und danach mit der Hälfte
ihrer Nennkapazität wieder geladen.
Danach wird eine \gls{deis}-Messung mit vordefinierter C-Rate durchgeführt.
Wahrend der Messung wird die Zelle \si{5}{\%} ihrer Kapazität geladen.
Diese Ladungsmenge wird ihr im nächsten Schritt wieder abgezogen.
Nach einer kurzen Pause beginnt die nächste \gls{deis}-Messung mit einer anderen C-Rate.
Als C-Raten werden \SI{0.25}{C}, \SI{0.33}{C}, \SI{0.5}{C}, \SI{1}{C}, \SI{2}{C}, \SI{3}{C} verwendet, um ein
breites Spektrum abzudecken.
Auf dem Setup von Biologic wird die Prüfung noch einmal wiederholt, wobei die Impedanzmessung bei \SI{2}{C}
und \SI{3}{C} auf zwei Messungen aufgeteilt, um die Zelle wieder abkühlen zu lassen und Temperatureinflüsse gering zu halten.

\begin{figure} [h]
    \hspace{3cm}% Abstand nach rechts verschieben
    \centering
    \begin{tikzpicture}[node distance=0.75cm and 5]

    % Nodes
    \node (dis1) [diamond] {Entladen auf \SI{0}{\%} SOC};
    \node (cha1) [diamond, below=of dis1] {Laden auf \SI{50}{\%} SOC};
    \node (deis) [rectangle, below=of cha1] {DEIS-Messung};
    \node (dis2) [rectangle, below=of deis] {Entladen};

    % Arrows
    \draw [arrow] (dis1) -- (cha1);
    \draw [arrow] (cha1) -- (deis);
    \draw [arrow] (deis) -- (dis2);
    \draw [arrow] (dis2) to[out=0, in=0, looseness=1.5]
        node[arrowlabel] {\shortstack{5-Mal wiederholen mit \\ unterschiedlichen C-Raten}} (deis);

    \end{tikzpicture}
    \caption{Prüfplan für die Untersuchung von Formierungseffekten}\label{fig:pruefplan}
    \label{fig:pruefplan-basy}
\end{figure}

\subsection{Berechnung der Impedanz}
Das Messgerät, die Zelle und der Zyklisierer befinden sich in einer parallelschalten~\ref{fig: schaltplan-basy}.
Daraus ergibt sich die Formel~\ref{eq:imp2} für die Impedanz des Zyklisieres.

\begin{figure}[H]
    \centering
    \input{figures/ecd/ecd_basy}
    \caption{Erstazschaltbild des Prüfaufbaus mit BaSyTec XCTS, Inspectrum und Batteriezelle}\label{fig:schaltbild}
    \label{fig: schaltplan-basy}
\end{figure}

\begin{equation}
    \frac{1}{Z_{ges}} = \frac{1}{Z_{Zelle}} +  \frac{1}{Z_{XCTS}}
\label{eq:imp1}
\end{equation}
\begin{equation}
    Z_{XCTS} = \frac{Z_{ges} Z_{Zelle}}{Z_{ges} - Z_{Zelle}}
\label{eq:imp2}
\end{equation}

