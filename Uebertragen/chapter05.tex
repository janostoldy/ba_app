\chapter{Ergebnisse}\label{ch:ergebnisse}%
\section{Formierungseinflüsse auf Impedanz}\label{sec:form}

\section{Impedanz von Basytec}\label{sec:impedanz-von-basytec}
Abbildung zeigt, dass es einen eindeutigen Unterschied zwischen Impedanzmessungen auf den beiden Setups gibt.
Dies erklärt die Abweichung anderer Messungen auf diesen Setups.
Der Vergleich von Impedanzmessungen der beiden Messaufbauten zeigt, dass bei Basytec die Stromstärke nur einen sehr
geringen Einfluss auf die Messungen hat, für Biologic ist der Einfluss groß~\ref{fig:vgl_bio_basy_c_rate}.
Dies liegt zum Teil an der selbsterwärmung der Zellen, da die Messung auf Biologic deutlich länger dauern.
Eine weite Folge der längeren Messung ist, dass soch der \gls{soc} besonders bei hohen C-Raten während der Messung
verändert.
Allerdings sind die Unterschiede auch bei niedrigen C-Raten deutlich erkennbar, wo noch keine nennenswerte
Selbsterwärmung stattfindet und sich der \gls{soc} der Zelle während der Messung nur gering verändert.
Auch beim


\begin{figure} [H]
    \centering
    \fontsize{10pt}{10pt}
    \centering
    \newcommand\svgwidth{1\linewidth}
    \input{figures/plots/c-raten-vgl.pdf_tex}
    \caption{Vergleich der Niquist-Kurve für verschiedene C-Raten auf beiden Setups}
    \label{fig:vgl_bio_basy_c_rate}
\end{figure}
